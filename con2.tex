\documentclass[a4paper,12pt]{article}
\usepackage{amsmath}
\begin{document}
\title{homework 2}
\author{HuJinghui (11061191)}
\date{\today}
\maketitle
\noindent
1) Solution,
$$(N, M) = (13, 5)$$
$$answer = 39$$
$$(N, M) = (13, 7)$$
$$answer = 37$$
$$(N, M) = (19, 5)$$
$$answer = 73$$
$$(N, M) = (19, 7)$$
$$answer = 59$$
$$(N, M) = (23, 5)$$
$$answer = 101$$
$$(N, M) = (23, 7)$$
$$answer = 75$$
\newline
2) The minimum amount of Hanoi problem is $T_n=2^n-1$, and I think sub-minimum
answer $T^{'}_{n}=T_n-1$ is correct. I will give a possible way to 
implemnet it. Now we have 3 pegs $(p_1, p_2, p_3)$, in the minimum solution, 
the last step is to move the smallest disk to $p_3$, instead of this step,
we can move the last disk to another empty peg, then move the last disk
to $p_3$.
\newline
\clearpage
\noindent
3) if break Lucas rule once \\
~\newline
\begin{tabular}{c|cc}
Disks & Break & No breaks\\ \hline
1 & 1 & 1\\
2 & 3 & 3\\
3 & 5 & 7\\
4 & 9 & 15\\
5 & 17 & 31\\
... & ... & ...
\end{tabular}
~\newline
if break Lucas rule once, the answer is $2^{n-1}+1$.\\
Suppose we have 3 pegs $(p_1, p_2, p_3)$, firstly, move $n-2$ disks to $p_2$,
using $2^{n-2}-1$ steps. Secondly, move the topest disk on $p_1$ to $p_2$ (
breaking Lucas rule). then, move the last disk to $p_3$ and move the topest disk
on $p_2$ to $p_3$, using $3$ steps. Finally, move all the disks on $p_2$ to
$p_3$, using $2^{n-2}-1$ steps. Totally, we'll use
\[
(2^{n-2}-1) \times 2 + 3 = 2^{n-1}+1
\]
steps.
\end{document}
